\hypertarget{_fortran_format_8py_source}{}\section{Fortran\+Format.\+py}
\label{_fortran_format_8py_source}\index{pyneb/utils/\+Fortran\+Format.\+py@{pyneb/utils/\+Fortran\+Format.\+py}}

\begin{DoxyCode}
\hypertarget{_fortran_format_8py_source_l00001}{}\hyperlink{namespacepyneb_1_1utils_1_1_fortran_format}{00001} \textcolor{comment}{# This module defines a class that handles I/O using}
00002 \textcolor{comment}{# Fortran-compatible format specifications.}
00003 \textcolor{comment}{#}
00004 \textcolor{comment}{#}
00005 \textcolor{comment}{# Warning: Fortran formatting is a complex business and I don't}
00006 \textcolor{comment}{# claim that this module works for anything complicated. It knows}
00007 \textcolor{comment}{# only the most frequent formatting options. Known limitations:}
00008 \textcolor{comment}{#}
00009 \textcolor{comment}{# 1) Only A, D, E, F, G, I, and X formats are supported (plus string constants}
00010 \textcolor{comment}{#    for output).}
00011 \textcolor{comment}{# 2) No direct support for complex numbers. You have to split them into}
00012 \textcolor{comment}{#    real and imaginary parts before output, and for input you get}
00013 \textcolor{comment}{#    two float numbers anyway.}
00014 \textcolor{comment}{# 3) No overflow check. If an output field gets too large, it will}
00015 \textcolor{comment}{#    take more space, instead of being replaced by stars.}
00016 \textcolor{comment}{#}
00017 \textcolor{comment}{#}
00018 \textcolor{comment}{# Written by Konrad Hinsen <hinsen@cnrs-orleans.fr>}
00019 \textcolor{comment}{# With contributions from Andreas Prlic <andreas@came.sbg.ac.at>}
00020 \textcolor{comment}{# last revision: 2006-6-23}
00021 \textcolor{comment}{#}
00022 
00023 \textcolor{stringliteral}{"""}
00024 \textcolor{stringliteral}{Fortran-style formatted input/output}
00025 \textcolor{stringliteral}{}
00026 \textcolor{stringliteral}{This module provides two classes that aid in reading and writing}
00027 \textcolor{stringliteral}{Fortran-formatted text files.}
00028 \textcolor{stringliteral}{}
00029 \textcolor{stringliteral}{Examples::}
00030 \textcolor{stringliteral}{}
00031 \textcolor{stringliteral}{  Input::}
00032 \textcolor{stringliteral}{}
00033 \textcolor{stringliteral}{    >>>s = '   59999'}
00034 \textcolor{stringliteral}{    >>>format = FortranFormat('2I4')}
00035 \textcolor{stringliteral}{    >>>line = FortranLine(s, format)}
00036 \textcolor{stringliteral}{    >>>print line[0]}
00037 \textcolor{stringliteral}{    >>>print line[1]}
00038 \textcolor{stringliteral}{}
00039 \textcolor{stringliteral}{  prints::}
00040 \textcolor{stringliteral}{}
00041 \textcolor{stringliteral}{    >>>5}
00042 \textcolor{stringliteral}{    >>>9999}
00043 \textcolor{stringliteral}{}
00044 \textcolor{stringliteral}{}
00045 \textcolor{stringliteral}{  Output::}
00046 \textcolor{stringliteral}{}
00047 \textcolor{stringliteral}{    >>>format = FortranFormat('2D15.5')}
00048 \textcolor{stringliteral}{    >>>line = FortranLine([3.1415926, 2.71828], format)}
00049 \textcolor{stringliteral}{    >>>print str(line)}
00050 \textcolor{stringliteral}{}
00051 \textcolor{stringliteral}{  prints::}
00052 \textcolor{stringliteral}{}
00053 \textcolor{stringliteral}{    '3.14159D+00    2.71828D+00'}
00054 \textcolor{stringliteral}{"""}
00055 
00056 \textcolor{keyword}{import} string
00057 
00058 \textcolor{comment}{#}
00059 \textcolor{comment}{# The class FortranLine represents a single line of input/output,}
00060 \textcolor{comment}{# which can be accessed as text or as a list of items.}
00061 \textcolor{comment}{#}
\hypertarget{_fortran_format_8py_source_l00062}{}\hyperlink{classpyneb_1_1utils_1_1_fortran_format_1_1_fortran_line}{00062} \textcolor{keyword}{class }\hyperlink{classpyneb_1_1utils_1_1_fortran_format_1_1_fortran_line}{FortranLine}:
00063 
00064     \textcolor{stringliteral}{"""Fortran-style record in formatted files}
00065 \textcolor{stringliteral}{}
00066 \textcolor{stringliteral}{    FortranLine objects represent the content of one record of a}
00067 \textcolor{stringliteral}{    Fortran-style formatted file. Indexing yields the contents as}
00068 \textcolor{stringliteral}{    Python objects, whereas transformation to a string (using the}
00069 \textcolor{stringliteral}{    built-in function 'str') yields the text representation.}
00070 \textcolor{stringliteral}{}
00071 \textcolor{stringliteral}{    Restrictions:}
00072 \textcolor{stringliteral}{}
00073 \textcolor{stringliteral}{      1. Only A, D, E, F, G, I, and X formats are supported (plus string}
00074 \textcolor{stringliteral}{         constants for output).}
00075 \textcolor{stringliteral}{}
00076 \textcolor{stringliteral}{      2. No direct support for complex numbers; they must be split into}
00077 \textcolor{stringliteral}{         real and imaginary parts before output.}
00078 \textcolor{stringliteral}{}
00079 \textcolor{stringliteral}{      3. No overflow check. If an output field gets too large, it will}
00080 \textcolor{stringliteral}{         take more space, instead of being replaced by stars according}
00081 \textcolor{stringliteral}{         to Fortran conventions.}
00082 \textcolor{stringliteral}{    """}
00083 
\hypertarget{_fortran_format_8py_source_l00084}{}\hyperlink{classpyneb_1_1utils_1_1_fortran_format_1_1_fortran_line_a292ba8e3d93bd751d6d13fdc1671ccb2}{00084}     \textcolor{keyword}{def }\hyperlink{classpyneb_1_1utils_1_1_fortran_format_1_1_fortran_line_a292ba8e3d93bd751d6d13fdc1671ccb2}{\_\_init\_\_}(self, line, format, length = 80):
00085         \textcolor{stringliteral}{"""}
00086 \textcolor{stringliteral}{        @param data: either a sequence of Python objects, or a string}
00087 \textcolor{stringliteral}{                     formatted according to Fortran rules}
00088 \textcolor{stringliteral}{}
00089 \textcolor{stringliteral}{        @param format: either a Fortran-style format string, or a}
00090 \textcolor{stringliteral}{                       L\{FortranFormat\} object. A FortranFormat should}
00091 \textcolor{stringliteral}{                       be used when the same format string is used repeatedly,}
00092 \textcolor{stringliteral}{                       because then the rather slow parsing of the string}
00093 \textcolor{stringliteral}{                       is performed only once.}
00094 \textcolor{stringliteral}{}
00095 \textcolor{stringliteral}{        @param length: the length of the Fortran record. This is relevant}
00096 \textcolor{stringliteral}{                       only when data is a string; this string is then}
00097 \textcolor{stringliteral}{                       extended by spaces to have the indicated length.}
00098 \textcolor{stringliteral}{                       The default value of 80 is almost always correct.}
00099 \textcolor{stringliteral}{        """}
00100         \textcolor{keywordflow}{if} type(line) == type(\textcolor{stringliteral}{''}):
\hypertarget{_fortran_format_8py_source_l00101}{}\hyperlink{classpyneb_1_1utils_1_1_fortran_format_1_1_fortran_line_a70d4893b8dd8ae61297b1d3e4b8bc612}{00101}             self.\hyperlink{classpyneb_1_1utils_1_1_fortran_format_1_1_fortran_line_a70d4893b8dd8ae61297b1d3e4b8bc612}{text} = line
\hypertarget{_fortran_format_8py_source_l00102}{}\hyperlink{classpyneb_1_1utils_1_1_fortran_format_1_1_fortran_line_a5976b8e1d4375a2ea62b9359bcf84697}{00102}             self.\hyperlink{classpyneb_1_1utils_1_1_fortran_format_1_1_fortran_line_a5976b8e1d4375a2ea62b9359bcf84697}{data} = \textcolor{keywordtype}{None}
00103         \textcolor{keywordflow}{else}:
00104             self.\hyperlink{classpyneb_1_1utils_1_1_fortran_format_1_1_fortran_line_a70d4893b8dd8ae61297b1d3e4b8bc612}{text} = \textcolor{keywordtype}{None}
00105             self.\hyperlink{classpyneb_1_1utils_1_1_fortran_format_1_1_fortran_line_a5976b8e1d4375a2ea62b9359bcf84697}{data} = line
00106         \textcolor{keywordflow}{if} type(format) == type(\textcolor{stringliteral}{''}):
\hypertarget{_fortran_format_8py_source_l00107}{}\hyperlink{classpyneb_1_1utils_1_1_fortran_format_1_1_fortran_line_ae97c8744bdb9817ec8a9c58b7a6bfe20}{00107}             self.\hyperlink{classpyneb_1_1utils_1_1_fortran_format_1_1_fortran_line_ae97c8744bdb9817ec8a9c58b7a6bfe20}{format} = \hyperlink{classpyneb_1_1utils_1_1_fortran_format_1_1_fortran_format}{FortranFormat}(format)
00108         \textcolor{keywordflow}{else}:
00109             self.\hyperlink{classpyneb_1_1utils_1_1_fortran_format_1_1_fortran_line_ae97c8744bdb9817ec8a9c58b7a6bfe20}{format} = format
\hypertarget{_fortran_format_8py_source_l00110}{}\hyperlink{classpyneb_1_1utils_1_1_fortran_format_1_1_fortran_line_a4cd0f5ee28f8250f2c6fbbb9ff890e0a}{00110}         self.\hyperlink{classpyneb_1_1utils_1_1_fortran_format_1_1_fortran_line_a4cd0f5ee28f8250f2c6fbbb9ff890e0a}{length} = length
00111         \textcolor{keywordflow}{if} self.\hyperlink{classpyneb_1_1utils_1_1_fortran_format_1_1_fortran_line_a70d4893b8dd8ae61297b1d3e4b8bc612}{text} \textcolor{keywordflow}{is} \textcolor{keywordtype}{None}:
00112             self.\hyperlink{classpyneb_1_1utils_1_1_fortran_format_1_1_fortran_line_ad05b9dd81ba899bde2235ae9409a6f38}{\_output}()
00113         \textcolor{keywordflow}{if} self.\hyperlink{classpyneb_1_1utils_1_1_fortran_format_1_1_fortran_line_a5976b8e1d4375a2ea62b9359bcf84697}{data} \textcolor{keywordflow}{is} \textcolor{keywordtype}{None}:
00114             self.\hyperlink{classpyneb_1_1utils_1_1_fortran_format_1_1_fortran_line_a578ada89c0a7d1e519ba15fca4888073}{\_input}()
00115 
\hypertarget{_fortran_format_8py_source_l00116}{}\hyperlink{classpyneb_1_1utils_1_1_fortran_format_1_1_fortran_line_ac8b694644f83c524f84a3f093c5309e3}{00116}     \textcolor{keyword}{def }\hyperlink{classpyneb_1_1utils_1_1_fortran_format_1_1_fortran_line_ac8b694644f83c524f84a3f093c5309e3}{\_\_len\_\_}(self):
00117         \textcolor{stringliteral}{"""}
00118 \textcolor{stringliteral}{        @returns: the number of data elements in the record}
00119 \textcolor{stringliteral}{        @rtype: C\{int\}}
00120 \textcolor{stringliteral}{        """}
00121         \textcolor{keywordflow}{return} len(self.\hyperlink{classpyneb_1_1utils_1_1_fortran_format_1_1_fortran_line_a5976b8e1d4375a2ea62b9359bcf84697}{data})
00122 
\hypertarget{_fortran_format_8py_source_l00123}{}\hyperlink{classpyneb_1_1utils_1_1_fortran_format_1_1_fortran_line_a1fc8a0bef1841d12d258cff06e2c5c4d}{00123}     \textcolor{keyword}{def }\hyperlink{classpyneb_1_1utils_1_1_fortran_format_1_1_fortran_line_a1fc8a0bef1841d12d258cff06e2c5c4d}{\_\_getitem\_\_}(self, i):
00124         \textcolor{stringliteral}{"""}
00125 \textcolor{stringliteral}{        @param i: index}
00126 \textcolor{stringliteral}{        @type i: C\{int\}}
00127 \textcolor{stringliteral}{        @returns: the ith data element}
00128 \textcolor{stringliteral}{        """}
00129         \textcolor{keywordflow}{return} self.\hyperlink{classpyneb_1_1utils_1_1_fortran_format_1_1_fortran_line_a5976b8e1d4375a2ea62b9359bcf84697}{data}[i]
00130 
\hypertarget{_fortran_format_8py_source_l00131}{}\hyperlink{classpyneb_1_1utils_1_1_fortran_format_1_1_fortran_line_a623c152bf4892c671290b6f9dcb0fee4}{00131}     \textcolor{keyword}{def }\hyperlink{classpyneb_1_1utils_1_1_fortran_format_1_1_fortran_line_a623c152bf4892c671290b6f9dcb0fee4}{\_\_getslice\_\_}(self, i, j):
00132         \textcolor{stringliteral}{"""}
00133 \textcolor{stringliteral}{        @param i: start index}
00134 \textcolor{stringliteral}{        @type i: C\{int\}}
00135 \textcolor{stringliteral}{        @param j: end index}
00136 \textcolor{stringliteral}{        @type j: C\{int\}}
00137 \textcolor{stringliteral}{        @returns: a list containing the ith to jth data elements}
00138 \textcolor{stringliteral}{        """}
00139         \textcolor{keywordflow}{return} self.\hyperlink{classpyneb_1_1utils_1_1_fortran_format_1_1_fortran_line_a5976b8e1d4375a2ea62b9359bcf84697}{data}[i:j]
00140 
\hypertarget{_fortran_format_8py_source_l00141}{}\hyperlink{classpyneb_1_1utils_1_1_fortran_format_1_1_fortran_line_a02ccdba49c33daca4eb4187d3aa23057}{00141}     \textcolor{keyword}{def }\hyperlink{classpyneb_1_1utils_1_1_fortran_format_1_1_fortran_line_a02ccdba49c33daca4eb4187d3aa23057}{\_\_str\_\_}(self):
00142         \textcolor{stringliteral}{"""}
00143 \textcolor{stringliteral}{        @returns: a Fortran-formatted text representation of the data record}
00144 \textcolor{stringliteral}{        @rtype: C\{str\}}
00145 \textcolor{stringliteral}{        """}
00146         \textcolor{keywordflow}{return} self.\hyperlink{classpyneb_1_1utils_1_1_fortran_format_1_1_fortran_line_a70d4893b8dd8ae61297b1d3e4b8bc612}{text}
00147 
\hypertarget{_fortran_format_8py_source_l00148}{}\hyperlink{classpyneb_1_1utils_1_1_fortran_format_1_1_fortran_line_aaa590d7811b69dd963ffdc5510787101}{00148}     \textcolor{keyword}{def }\hyperlink{classpyneb_1_1utils_1_1_fortran_format_1_1_fortran_line_aaa590d7811b69dd963ffdc5510787101}{isBlank}(self):
00149         \textcolor{stringliteral}{"""}
00150 \textcolor{stringliteral}{        @returns: C\{True\} if the line contains only whitespace}
00151 \textcolor{stringliteral}{        @rtype: C\{bool\}}
00152 \textcolor{stringliteral}{        """}
00153         \textcolor{keywordflow}{return} len(string.strip(self.\hyperlink{classpyneb_1_1utils_1_1_fortran_format_1_1_fortran_line_a70d4893b8dd8ae61297b1d3e4b8bc612}{text})) == 0
00154 
\hypertarget{_fortran_format_8py_source_l00155}{}\hyperlink{classpyneb_1_1utils_1_1_fortran_format_1_1_fortran_line_a578ada89c0a7d1e519ba15fca4888073}{00155}     \textcolor{keyword}{def }\hyperlink{classpyneb_1_1utils_1_1_fortran_format_1_1_fortran_line_a578ada89c0a7d1e519ba15fca4888073}{\_input}(self):
00156         text = self.\hyperlink{classpyneb_1_1utils_1_1_fortran_format_1_1_fortran_line_a70d4893b8dd8ae61297b1d3e4b8bc612}{text}
00157         \textcolor{keywordflow}{if} len(text) < self.\hyperlink{classpyneb_1_1utils_1_1_fortran_format_1_1_fortran_line_a4cd0f5ee28f8250f2c6fbbb9ff890e0a}{length}: text = text + (self.\hyperlink{classpyneb_1_1utils_1_1_fortran_format_1_1_fortran_line_a4cd0f5ee28f8250f2c6fbbb9ff890e0a}{length}-len(text))*\textcolor{stringliteral}{' '}
00158         self.\hyperlink{classpyneb_1_1utils_1_1_fortran_format_1_1_fortran_line_a5976b8e1d4375a2ea62b9359bcf84697}{data} = []
00159         \textcolor{keywordflow}{for} field \textcolor{keywordflow}{in} self.\hyperlink{classpyneb_1_1utils_1_1_fortran_format_1_1_fortran_line_ae97c8744bdb9817ec8a9c58b7a6bfe20}{format}:
00160             l = field[1]
00161             s = text[:l]
00162             text = text[l:]
00163             type = field[0]
00164             value = \textcolor{keywordtype}{None}
00165             \textcolor{keywordflow}{if} type == \textcolor{stringliteral}{'A'}:
00166                 value = s
00167             \textcolor{keywordflow}{elif} type == \textcolor{stringliteral}{'I'}:
00168                 s = string.strip(s)
00169                 \textcolor{keywordflow}{if} len(s) == 0:
00170                     value = 0
00171                 \textcolor{keywordflow}{else}:
00172                     \textcolor{comment}{# by AP}
00173                     \textcolor{comment}{# sometimes a line does not match to expected format,}
00174                     \textcolor{comment}{# e.g.: pdb2myd.ent.Z chain: - model: 0 : CONECT*****}
00175                     \textcolor{comment}{# catch this and set value to None}
00176                     \textcolor{keywordflow}{try}:
00177                         value = string.atoi(s)
00178                     \textcolor{keywordflow}{except}:
00179                         value = \textcolor{keywordtype}{None}
00180             \textcolor{keywordflow}{elif} type == \textcolor{stringliteral}{'D'} \textcolor{keywordflow}{or} type == \textcolor{stringliteral}{'E'} \textcolor{keywordflow}{or} type == \textcolor{stringliteral}{'F'} \textcolor{keywordflow}{or} type == \textcolor{stringliteral}{'G'}:
00181                 s = string.lower(string.strip(s))
00182                 n = string.find(s, \textcolor{stringliteral}{'d'})
00183                 \textcolor{keywordflow}{if} n >= 0:
00184                     s = s[:n] + \textcolor{stringliteral}{'e'} + s[n+1:]
00185                 \textcolor{keywordflow}{if} len(s) == 0:
00186                     value = 0.
00187                 \textcolor{keywordflow}{else}:
00188                     \textcolor{keywordflow}{try}:
00189                         value = string.atof(s)
00190                     \textcolor{keywordflow}{except}:
00191                         value = \textcolor{keywordtype}{None}
00192             \textcolor{keywordflow}{if} value \textcolor{keywordflow}{is} \textcolor{keywordflow}{not} \textcolor{keywordtype}{None}:
00193                 self.data.append(value)
00194 
\hypertarget{_fortran_format_8py_source_l00195}{}\hyperlink{classpyneb_1_1utils_1_1_fortran_format_1_1_fortran_line_ad05b9dd81ba899bde2235ae9409a6f38}{00195}     \textcolor{keyword}{def }\hyperlink{classpyneb_1_1utils_1_1_fortran_format_1_1_fortran_line_ad05b9dd81ba899bde2235ae9409a6f38}{\_output}(self):
00196         data = self.\hyperlink{classpyneb_1_1utils_1_1_fortran_format_1_1_fortran_line_a5976b8e1d4375a2ea62b9359bcf84697}{data}
00197         self.\hyperlink{classpyneb_1_1utils_1_1_fortran_format_1_1_fortran_line_a70d4893b8dd8ae61297b1d3e4b8bc612}{text} = \textcolor{stringliteral}{''}
00198         \textcolor{keywordflow}{for} field \textcolor{keywordflow}{in} self.\hyperlink{classpyneb_1_1utils_1_1_fortran_format_1_1_fortran_line_ae97c8744bdb9817ec8a9c58b7a6bfe20}{format}:
00199             type = field[0]
00200             \textcolor{keywordflow}{if} type == \textcolor{stringliteral}{"'"}:
00201                 self.\hyperlink{classpyneb_1_1utils_1_1_fortran_format_1_1_fortran_line_a70d4893b8dd8ae61297b1d3e4b8bc612}{text} = self.\hyperlink{classpyneb_1_1utils_1_1_fortran_format_1_1_fortran_line_a70d4893b8dd8ae61297b1d3e4b8bc612}{text} + field[1]
00202             \textcolor{keywordflow}{elif} type == \textcolor{stringliteral}{'X'}:
00203                 self.\hyperlink{classpyneb_1_1utils_1_1_fortran_format_1_1_fortran_line_a70d4893b8dd8ae61297b1d3e4b8bc612}{text} = self.\hyperlink{classpyneb_1_1utils_1_1_fortran_format_1_1_fortran_line_a70d4893b8dd8ae61297b1d3e4b8bc612}{text} + field[1]*\textcolor{stringliteral}{' '}
00204             \textcolor{keywordflow}{else}: \textcolor{comment}{# fields that use input data}
00205                 length = field[1]
00206                 \textcolor{keywordflow}{if} len(field) > 2: fraction = field[2]
00207                 value = data[0]
00208                 data = data[1:]
00209                 \textcolor{keywordflow}{if} type == \textcolor{stringliteral}{'A'}:
00210                     self.\hyperlink{classpyneb_1_1utils_1_1_fortran_format_1_1_fortran_line_a70d4893b8dd8ae61297b1d3e4b8bc612}{text} = self.\hyperlink{classpyneb_1_1utils_1_1_fortran_format_1_1_fortran_line_a70d4893b8dd8ae61297b1d3e4b8bc612}{text} + (value+length*\textcolor{stringliteral}{' '})[:length]
00211                 \textcolor{keywordflow}{else}: \textcolor{comment}{# numeric fields}
00212                     \textcolor{keywordflow}{if} value \textcolor{keywordflow}{is} \textcolor{keywordtype}{None}:
00213                         s = \textcolor{stringliteral}{''}
00214                     \textcolor{keywordflow}{elif} type == \textcolor{stringliteral}{'I'}:
00215                         s = repr(value)
00216                     \textcolor{keywordflow}{elif} type == \textcolor{stringliteral}{'D'}:
00217                         s = (\textcolor{stringliteral}{'%'}+repr(length)+\textcolor{stringliteral}{'.'}+repr(fraction)+\textcolor{stringliteral}{'e'}) % value
00218                         n = string.find(s, \textcolor{stringliteral}{'e'})
00219                         s = s[:n] + \textcolor{stringliteral}{'D'} + s[n+1:]
00220                     \textcolor{keywordflow}{elif} type == \textcolor{stringliteral}{'E'}:
00221                         s = (\textcolor{stringliteral}{'%'}+repr(length)+\textcolor{stringliteral}{'.'}+ repr(fraction)+\textcolor{stringliteral}{'e'}) % value
00222                     \textcolor{keywordflow}{elif} type == \textcolor{stringliteral}{'F'}:
00223                         s = (\textcolor{stringliteral}{'%'}+repr(length)+\textcolor{stringliteral}{'.'}+repr(fraction)+\textcolor{stringliteral}{'f'}) % value
00224                     \textcolor{keywordflow}{elif} type == \textcolor{stringliteral}{'G'}:
00225                         s = (\textcolor{stringliteral}{'%'}+repr(length)+\textcolor{stringliteral}{'.'}+repr(fraction)+\textcolor{stringliteral}{'g'}) % value
00226                     \textcolor{keywordflow}{else}:
00227                         \textcolor{keywordflow}{raise} ValueError(\textcolor{stringliteral}{'Not yet implemented'})
00228                     s = string.upper(s)
00229                     self.\hyperlink{classpyneb_1_1utils_1_1_fortran_format_1_1_fortran_line_a70d4893b8dd8ae61297b1d3e4b8bc612}{text} = self.\hyperlink{classpyneb_1_1utils_1_1_fortran_format_1_1_fortran_line_a70d4893b8dd8ae61297b1d3e4b8bc612}{text} + ((length*\textcolor{stringliteral}{' '})+s)[-length:]
00230         self.\hyperlink{classpyneb_1_1utils_1_1_fortran_format_1_1_fortran_line_a70d4893b8dd8ae61297b1d3e4b8bc612}{text} = string.rstrip(self.\hyperlink{classpyneb_1_1utils_1_1_fortran_format_1_1_fortran_line_a70d4893b8dd8ae61297b1d3e4b8bc612}{text})
00231 
00232 \textcolor{comment}{#}
00233 \textcolor{comment}{# The class FortranFormat represents a format specification.}
00234 \textcolor{comment}{# It ought to work for correct specifications, but there is}
00235 \textcolor{comment}{# little error checking.}
00236 \textcolor{comment}{#}
\hypertarget{_fortran_format_8py_source_l00237}{}\hyperlink{classpyneb_1_1utils_1_1_fortran_format_1_1_fortran_format}{00237} \textcolor{keyword}{class }\hyperlink{classpyneb_1_1utils_1_1_fortran_format_1_1_fortran_format}{FortranFormat}:
00238 
00239     \textcolor{stringliteral}{"""}
00240 \textcolor{stringliteral}{    Parsed Fortran-style format string}
00241 \textcolor{stringliteral}{}
00242 \textcolor{stringliteral}{    FortranFormat objects can be used as arguments when constructing}
00243 \textcolor{stringliteral}{    FortranLine objects instead of the plain format string. If a}
00244 \textcolor{stringliteral}{    format string is used more than once, embedding it into a FortranFormat}
00245 \textcolor{stringliteral}{    object has the advantage that the format string is parsed only once.}
00246 \textcolor{stringliteral}{    """}
00247 
\hypertarget{_fortran_format_8py_source_l00248}{}\hyperlink{classpyneb_1_1utils_1_1_fortran_format_1_1_fortran_format_a7fc0109f1fa73801ad3795463473e41a}{00248}     \textcolor{keyword}{def }\hyperlink{classpyneb_1_1utils_1_1_fortran_format_1_1_fortran_format_a7fc0109f1fa73801ad3795463473e41a}{\_\_init\_\_}(self, format, nested = False):
00249         \textcolor{stringliteral}{"""}
00250 \textcolor{stringliteral}{        @param format: a Fortran format specification}
00251 \textcolor{stringliteral}{        @type format: C\{str\}}
00252 \textcolor{stringliteral}{        @param nested: I\{for internal use\}}
00253 \textcolor{stringliteral}{        """}
00254         fields = []
00255         format = string.strip(format)
00256         \textcolor{keywordflow}{while} format \textcolor{keywordflow}{and} format[0] != \textcolor{stringliteral}{')'}:
00257             n = 0
00258             \textcolor{keywordflow}{while} format[0] \textcolor{keywordflow}{in} string.digits:
00259                 n = 10*n + string.atoi(format[0])
00260                 format = format[1:]
00261             \textcolor{keywordflow}{if} n == 0: n = 1
00262             type = string.upper(format[0])
00263             \textcolor{keywordflow}{if} type == \textcolor{stringliteral}{"'"}:
00264                 eof = string.find(format, \textcolor{stringliteral}{"'"}, 1)
00265                 text = format[1:eof]
00266                 format = format[eof+1:]
00267             \textcolor{keywordflow}{else}:
00268                 format = string.strip(format[1:])
00269             \textcolor{keywordflow}{if} type == \textcolor{stringliteral}{'('}:
00270                 subformat = \hyperlink{classpyneb_1_1utils_1_1_fortran_format_1_1_fortran_format}{FortranFormat}(format, 1)
00271                 fields = fields + n*subformat.fields
00272                 format = subformat.rest
00273                 eof = string.find(format, \textcolor{stringliteral}{','})
00274                 \textcolor{keywordflow}{if} eof >= 0:
00275                     format = format[eof+1:]
00276             \textcolor{keywordflow}{else}:
00277                 eof = string.find(format, \textcolor{stringliteral}{','})
00278                 \textcolor{keywordflow}{if} eof >= 0:
00279                     field = format[:eof]
00280                     format = format[eof+1:]
00281                 \textcolor{keywordflow}{else}:
00282                     eof = string.find(format, \textcolor{stringliteral}{')'})
00283                     \textcolor{keywordflow}{if} eof >= 0:
00284                         field = format[:eof]
00285                         format = format[eof+1:]
00286                     \textcolor{keywordflow}{else}:
00287                         field = format
00288                         format = \textcolor{stringliteral}{''}
00289                 \textcolor{keywordflow}{if} type == \textcolor{stringliteral}{"'"}:
00290                     field = (type, text)
00291                 \textcolor{keywordflow}{else}:
00292                     dot = string.find(field, \textcolor{stringliteral}{'.'})
00293                     \textcolor{keywordflow}{if} dot > 0:
00294                         length = string.atoi(field[:dot])
00295                         fraction = string.atoi(field[dot+1:])
00296                         field = (type, length, fraction)
00297                     \textcolor{keywordflow}{else}:
00298                         \textcolor{keywordflow}{if} field:
00299                             length = string.atoi(field)
00300                         \textcolor{keywordflow}{else}:
00301                             length = 1
00302                         field = (type, length)
00303                 fields = fields + n*[field]
\hypertarget{_fortran_format_8py_source_l00304}{}\hyperlink{classpyneb_1_1utils_1_1_fortran_format_1_1_fortran_format_ac8f962282ba9e35ccd8dad5feacd3813}{00304}         self.\hyperlink{classpyneb_1_1utils_1_1_fortran_format_1_1_fortran_format_ac8f962282ba9e35ccd8dad5feacd3813}{fields} = fields
00305         \textcolor{keywordflow}{if} nested:
\hypertarget{_fortran_format_8py_source_l00306}{}\hyperlink{classpyneb_1_1utils_1_1_fortran_format_1_1_fortran_format_a6567d6b3709b1db086972c2fe1855006}{00306}             self.\hyperlink{classpyneb_1_1utils_1_1_fortran_format_1_1_fortran_format_a6567d6b3709b1db086972c2fe1855006}{rest} = format
00307 
\hypertarget{_fortran_format_8py_source_l00308}{}\hyperlink{classpyneb_1_1utils_1_1_fortran_format_1_1_fortran_format_abd54b6a502a35a8e4dabcbcb0632cff7}{00308}     \textcolor{keyword}{def }\hyperlink{classpyneb_1_1utils_1_1_fortran_format_1_1_fortran_format_abd54b6a502a35a8e4dabcbcb0632cff7}{\_\_len\_\_}(self):
00309         \textcolor{keywordflow}{return} len(self.\hyperlink{classpyneb_1_1utils_1_1_fortran_format_1_1_fortran_format_ac8f962282ba9e35ccd8dad5feacd3813}{fields})
00310 
\hypertarget{_fortran_format_8py_source_l00311}{}\hyperlink{classpyneb_1_1utils_1_1_fortran_format_1_1_fortran_format_a8fc69af902245845b7fceba5e8f7974f}{00311}     \textcolor{keyword}{def }\hyperlink{classpyneb_1_1utils_1_1_fortran_format_1_1_fortran_format_a8fc69af902245845b7fceba5e8f7974f}{\_\_getitem\_\_}(self, i):
00312         \textcolor{keywordflow}{return} self.\hyperlink{classpyneb_1_1utils_1_1_fortran_format_1_1_fortran_format_ac8f962282ba9e35ccd8dad5feacd3813}{fields}[i]
00313 
00314 
00315 \textcolor{comment}{# Test code}
00316 
00317 \textcolor{keywordflow}{if} \_\_name\_\_ == \textcolor{stringliteral}{'\_\_main\_\_'}:
\hypertarget{_fortran_format_8py_source_l00318}{}\hyperlink{namespacepyneb_1_1utils_1_1_fortran_format_af94352584eced016c14d524330879115}{00318}     f = \hyperlink{classpyneb_1_1utils_1_1_fortran_format_1_1_fortran_format}{FortranFormat}(\textcolor{stringliteral}{"'!!',D10.3,F10.3,G10.3,'!!'"})
\hypertarget{_fortran_format_8py_source_l00319}{}\hyperlink{namespacepyneb_1_1utils_1_1_fortran_format_a80ea829e31363f67a89c70955e194b93}{00319}     l = \hyperlink{classpyneb_1_1utils_1_1_fortran_format_1_1_fortran_line}{FortranLine}([1.5707963, 3.14159265358, 2.71828], f)
00320     print((str(l)))
00321     f = \hyperlink{classpyneb_1_1utils_1_1_fortran_format_1_1_fortran_format}{FortranFormat}(\textcolor{stringliteral}{"F12.0"})
00322     l = \hyperlink{classpyneb_1_1utils_1_1_fortran_format_1_1_fortran_line}{FortranLine}(\textcolor{stringliteral}{'2.1D2'}, f)
00323     print((l[0]))
\end{DoxyCode}

\hypertarget{stellar_8py}{\section{stellar.\-py}
\label{stellar_8py}\index{pyneb/utils/stellar.\-py@{pyneb/utils/stellar.\-py}}
}

\begin{DoxyCode}
\hypertarget{stellar_8py_source_l00001}{}\hyperlink{namespacepyneb_1_1utils_1_1stellar}{00001} \textcolor{comment}{#/opt/local/bin/python}
00002 \textcolor{comment}{# Compute stellar Zanstra temperature and stellar luminosity.}
00003 \textcolor{comment}{# Last update: 2013-Oct-30 by RAS}
00004 
00005 \textcolor{keyword}{import} numpy \textcolor{keyword}{as} np
00006 \textcolor{keyword}{import} scipy
00007 \textcolor{keyword}{import} scipy.optimize \textcolor{keyword}{as} op
00008 \textcolor{keyword}{from} scipy \textcolor{keyword}{import} constants \textcolor{keyword}{as} phy
00009 \textcolor{keyword}{from} scipy.integrate \textcolor{keyword}{import} quad
00010 
\hypertarget{stellar_8py_source_l00011}{}\hyperlink{namespacepyneb_1_1utils_1_1stellar_afb8d2a4b14477b7097da285535eafc12}{00011} WAV\_H\_IONIZ   = 911.7634
\hypertarget{stellar_8py_source_l00012}{}\hyperlink{namespacepyneb_1_1utils_1_1stellar_a1b0844f39c08a2b7289ad093042cf55e}{00012} WAV\_HE2\_IONIZ = WAV\_H\_IONIZ/4.
\hypertarget{stellar_8py_source_l00013}{}\hyperlink{namespacepyneb_1_1utils_1_1stellar_a54ebf26e2af1a957afac5ce6db64de30}{00013} ABMAG0     = 48.60
\hypertarget{stellar_8py_source_l00014}{}\hyperlink{namespacepyneb_1_1utils_1_1stellar_a53dc996c80c0bc5ee416b1fcbb8a6afc}{00014} STMAG0     = 21.10
00015 \textcolor{comment}{#FLUX\_MAG0 = 3.68e-9    # from Pottasch (1984), outdated}
\hypertarget{stellar_8py_source_l00016}{}\hyperlink{namespacepyneb_1_1utils_1_1stellar_ab52d6f24f3c66cb2d6ad7e52bd4b4cd6}{00016} FLUX\_MAG0  = 3.63e-9    \textcolor{comment}{# or pow(10,-STMAG0/2.5)}
\hypertarget{stellar_8py_source_l00017}{}\hyperlink{namespacepyneb_1_1utils_1_1stellar_ab677115958b6555a42963ed87adc1afc}{00017} MBOL\_Sun   = 4.75
00018 
00019 \textcolor{stringliteral}{'''Determine the stellar Zanstra temperature from a nebular H or He emission }
00020 \textcolor{stringliteral}{   line flux and the stellar flux density. The method follows }
00021 \textcolor{stringliteral}{       Pottasch, S. 1984, "Planetary Nebulae" (Dordrecht: D. Reidel)}
00022 \textcolor{stringliteral}{   with some modern updates. Also determine Bolometric correction and stellar }
00023 \textcolor{stringliteral}{   luminosity given temperature and distance. }
00024 \textcolor{stringliteral}{}
00025 \textcolor{stringliteral}{   Limitations: }
00026 \textcolor{stringliteral}{    - assumes Blackbody stellar SED}
00027 \textcolor{stringliteral}{    - no correction for interstellar extinction.}
00028 \textcolor{stringliteral}{'''}
00029 
\hypertarget{stellar_8py_source_l00030}{}\hyperlink{namespacepyneb_1_1utils_1_1stellar_a1028b99ddfc374c050254672a26abda9}{00030} \textcolor{keyword}{def }\hyperlink{namespacepyneb_1_1utils_1_1stellar_a1028b99ddfc374c050254672a26abda9}{logL}(V, D\_pc, T\_eff):
00031     \textcolor{stringliteral}{'''Calculate the stellar luminosity in Solar units, given an (unreddened) V }
00032 \textcolor{stringliteral}{       apparent magnitude, a distance (pc), and the stellar temperature (K).}
00033 \textcolor{stringliteral}{    '''}
00034     \textcolor{keywordflow}{return} -0.4*(V - 2.5*np.log10(D\_pc) + \hyperlink{namespacepyneb_1_1utils_1_1stellar_adb31507adc6104012e696b5d0b28b2cc}{BC}(T\_eff) - MBOL\_Sun)
00035 
00036 
\hypertarget{stellar_8py_source_l00037}{}\hyperlink{namespacepyneb_1_1utils_1_1stellar_adb31507adc6104012e696b5d0b28b2cc}{00037} \textcolor{keyword}{def }\hyperlink{namespacepyneb_1_1utils_1_1stellar_adb31507adc6104012e696b5d0b28b2cc}{BC}(T\_eff):
00038     \textcolor{stringliteral}{'''Bolometric correction for very hot stars, from Vacca, Garmany & Shull,}
00039 \textcolor{stringliteral}{       1996, ApJ, 460, 914.}
00040 \textcolor{stringliteral}{    '''}
00041     \textcolor{keywordflow}{return} 27.66 - 6.84*np.log10(T\_eff)
00042 \textcolor{comment}{#    return 28.46 - 7.08*np.log10(T\_eff) + 0.08*log\_g  # Fit incl. gravity}
00043 
\hypertarget{stellar_8py_source_l00044}{}\hyperlink{namespacepyneb_1_1utils_1_1stellar_adf2a28268f7e69307f7b6c47920bcbb2}{00044} \textcolor{keyword}{def }\hyperlink{namespacepyneb_1_1utils_1_1stellar_adf2a28268f7e69307f7b6c47920bcbb2}{integrand}(x):
00045     \textcolor{keywordflow}{return} x**2/(np.exp(x)-1)
00046 
\hypertarget{stellar_8py_source_l00047}{}\hyperlink{namespacepyneb_1_1utils_1_1stellar_ab168636a519c17879ce58df1188ddbe1}{00047} \textcolor{keyword}{def }\hyperlink{namespacepyneb_1_1utils_1_1stellar_ab168636a519c17879ce58df1188ddbe1}{G}(ion,T):
00048     \textcolor{stringliteral}{'''Compute integral over frequency for Planck function.}
00049 \textcolor{stringliteral}{       ion - Reference Ion ("H | HE')}
00050 \textcolor{stringliteral}{       T   - Stellar temperature (K)}
00051 \textcolor{stringliteral}{       See Pottasch (1984), p. 169, eq. VII-8 for details. }
00052 \textcolor{stringliteral}{    '''}
00053 
00054     \textcolor{keywordflow}{if} ion.upper()==\textcolor{stringliteral}{'H'}:
00055         nu = phy.c / (WAV\_H\_IONIZ*1.e-10)
00056     \textcolor{keywordflow}{elif} ion.upper()==\textcolor{stringliteral}{'HE2'}:
00057         nu = phy.c / (WAV\_HE2\_IONIZ*1.e-10)
00058     \textcolor{keywordflow}{else}:
00059         \textcolor{keywordflow}{print} \textcolor{stringliteral}{'Unrecognized ion: %s'} % ion
00060         \textcolor{keywordflow}{return} \textcolor{keywordtype}{None}
00061 
00062     \textcolor{keywordflow}{return} quad(integrand, phy.h*nu/(phy.k*T), np.Inf)[0]
00063 
00064 
\hypertarget{stellar_8py_source_l00065}{}\hyperlink{classpyneb_1_1utils_1_1stellar_1_1_zanstra}{00065} \textcolor{keyword}{class }\hyperlink{classpyneb_1_1utils_1_1stellar_1_1_zanstra}{Zanstra}():
\hypertarget{stellar_8py_source_l00066}{}\hyperlink{classpyneb_1_1utils_1_1stellar_1_1_zanstra_a2316eead01f32d876ea59a1b02a669c0}{00066}     \textcolor{keyword}{def }\hyperlink{classpyneb_1_1utils_1_1stellar_1_1_zanstra_a2316eead01f32d876ea59a1b02a669c0}{\_\_init\_\_}(self, ion, logFlux=-10., magStar=10.):
00067         s = ion.upper()
00068         \textcolor{keywordflow}{if} s \textcolor{keywordflow}{in} (\textcolor{stringliteral}{'H'},\textcolor{stringliteral}{'HE2'}):
\hypertarget{stellar_8py_source_l00069}{}\hyperlink{classpyneb_1_1utils_1_1stellar_1_1_zanstra_a48c1797d46d49841c8b3a9275679e3ae}{00069}             self.\hyperlink{classpyneb_1_1utils_1_1stellar_1_1_zanstra_a48c1797d46d49841c8b3a9275679e3ae}{ion} = s
00070         \textcolor{keywordflow}{else}:
00071             self.\hyperlink{classpyneb_1_1utils_1_1stellar_1_1_zanstra_a48c1797d46d49841c8b3a9275679e3ae}{ion} = \textcolor{stringliteral}{'H'}
\hypertarget{stellar_8py_source_l00072}{}\hyperlink{classpyneb_1_1utils_1_1stellar_1_1_zanstra_a8c4fddabe252ad4c539244ace4e79227}{00072}         self.\hyperlink{classpyneb_1_1utils_1_1stellar_1_1_zanstra_a8c4fddabe252ad4c539244ace4e79227}{logFlux} = logFlux
\hypertarget{stellar_8py_source_l00073}{}\hyperlink{classpyneb_1_1utils_1_1stellar_1_1_zanstra_af635b1b8ca29b1b956b91e93f63fdf76}{00073}         self.\hyperlink{classpyneb_1_1utils_1_1stellar_1_1_zanstra_af635b1b8ca29b1b956b91e93f63fdf76}{magStar} = magStar
00074 
00075         \textcolor{keywordflow}{if} self.\hyperlink{classpyneb_1_1utils_1_1stellar_1_1_zanstra_a48c1797d46d49841c8b3a9275679e3ae}{ion} == \textcolor{stringliteral}{'H'}:
\hypertarget{stellar_8py_source_l00076}{}\hyperlink{classpyneb_1_1utils_1_1stellar_1_1_zanstra_a8f04a2133b2acf4641a2ce08db7fc100}{00076}             self.\hyperlink{classpyneb_1_1utils_1_1stellar_1_1_zanstra_a8f04a2133b2acf4641a2ce08db7fc100}{ionScale} = 3.95e-11
00077         \textcolor{keywordflow}{elif} self.\hyperlink{classpyneb_1_1utils_1_1stellar_1_1_zanstra_a48c1797d46d49841c8b3a9275679e3ae}{ion} == \textcolor{stringliteral}{'HE2'}:
00078             self.\hyperlink{classpyneb_1_1utils_1_1stellar_1_1_zanstra_a8f04a2133b2acf4641a2ce08db7fc100}{ionScale} = 8.49e-11
00079 
00080 
\hypertarget{stellar_8py_source_l00081}{}\hyperlink{classpyneb_1_1utils_1_1stellar_1_1_zanstra_a33d4960cdb38f6a25972f20da8d146d1}{00081}     \textcolor{keyword}{def }\hyperlink{classpyneb_1_1utils_1_1stellar_1_1_zanstra_a33d4960cdb38f6a25972f20da8d146d1}{z\_ratio}(self, T):
00082         \textcolor{stringliteral}{'''Compute the ratio of the flux in a nebular emission line to the stellar }
00083 \textcolor{stringliteral}{           flux density for a given stellar (black-body) temperature T.}
00084 \textcolor{stringliteral}{        '''}
00085 
00086         \textcolor{keywordflow}{return} self.\hyperlink{classpyneb_1_1utils_1_1stellar_1_1_zanstra_a8f04a2133b2acf4641a2ce08db7fc100}{ionScale} * pow(T,3) * \hyperlink{namespacepyneb_1_1utils_1_1stellar_ab168636a519c17879ce58df1188ddbe1}{G}(self.\hyperlink{classpyneb_1_1utils_1_1stellar_1_1_zanstra_a48c1797d46d49841c8b3a9275679e3ae}{ion}, T) * (np.exp(2.665e+4/T) - 1)
00087 
00088 
\hypertarget{stellar_8py_source_l00089}{}\hyperlink{classpyneb_1_1utils_1_1stellar_1_1_zanstra_abb29e7f875d83a00e081726574eae202}{00089}     \textcolor{keyword}{def }\hyperlink{classpyneb_1_1utils_1_1stellar_1_1_zanstra_abb29e7f875d83a00e081726574eae202}{Func}(self, T):
00090         \textcolor{stringliteral}{'''Return the discriminant of the stellar Zanstra temperature function.}
00091 \textcolor{stringliteral}{            '''}
00092 
00093         r\_obs = pow(10,self.\hyperlink{classpyneb_1_1utils_1_1stellar_1_1_zanstra_a8c4fddabe252ad4c539244ace4e79227}{logFlux}) / (FLUX\_MAG0 * pow(10,-self.\hyperlink{classpyneb_1_1utils_1_1stellar_1_1_zanstra_af635b1b8ca29b1b956b91e93f63fdf76}{magStar}/2.5))
00094         \textcolor{keywordflow}{return} r\_obs - self.\hyperlink{classpyneb_1_1utils_1_1stellar_1_1_zanstra_a33d4960cdb38f6a25972f20da8d146d1}{z\_ratio}(T)
00095 
00096 
\hypertarget{stellar_8py_source_l00097}{}\hyperlink{classpyneb_1_1utils_1_1stellar_1_1_zanstra_adbf52e2c709bd164b518256b61789840}{00097}     \textcolor{keyword}{def }\hyperlink{classpyneb_1_1utils_1_1stellar_1_1_zanstra_adbf52e2c709bd164b518256b61789840}{Solve}(self):
00098         \textcolor{stringliteral}{'''Determine the stellar Zanstra temperature from a nebular H or He }
00099 \textcolor{stringliteral}{           emission line flux and the stellar flux density.}
00100 \textcolor{stringliteral}{        '''}
00101 
00102         guess = 5.e+4
00103         \textcolor{keywordflow}{return} op.broyden1(self.\hyperlink{classpyneb_1_1utils_1_1stellar_1_1_zanstra_abb29e7f875d83a00e081726574eae202}{Func}, guess)
\end{DoxyCode}
